\documentclass[fleqn]{article}

\usepackage[margin=1in]{geometry}
\usepackage{amsmath}
\usepackage{siunitx}
\usepackage{booktabs}
\usepackage{footnote}
\usepackage[showseconds=false, showzone=false]{datetime2}
\makesavenoteenv{tabular}

\usepackage{fancyhdr}
\pagestyle{fancy}

\fancyfoot[l]{Thomas Malthouse}
\fancyfoot[c]{\thepage}
\fancyfoot[r]{\DTMnow}

\begin{document}
    \section*{Using the AD5206 digital potentiometer with AC signals}

    \begin{itemize}
        \item The AD5206 behaves like six mechanical potentiometers.
        \item Each potentiometer has three terminals (A,B, and W), just like a mechanical.
        \item These potentiometers are (to the best of my knowledge) completely isolated---I did not observe any leakage between them.
        \item There are two power terminals on the chip: $V_{\text{SS}}$ and $V_{\text{DD}}$.
        \item These terminals are subject to the constraints that
        \begin{equation*}
            V_{\text{SS}} \leq 0 \leq V_{\text{DD}}
        \end{equation*}
        and
        \begin{equation*}
            |V_{\text{DD}}| + |V_{\text{SS}}| \leq 5.5V
        \end{equation*}
        \item The potentials of all three wiper terminals MUST be between $V_\text{SS}$ and $V_\text{DD}$:
        \begin{equation*}
            V_\text{SS} \leq \{A,B,W\} \leq V_\text{DD}
        \end{equation*}
        \item If this condition is not met, the response is nonlinear, and signals are distorted.
    \end{itemize}
    \subsection*{Monopolar Mode}
        These potentiometers are usually used in (and most online resources assume) monopolar mode
        \begin{itemize}
            \item In this mode, $V_{\text{SS}} = 0V = \text{GND}$, and $V_{\text{DD}} = 5V$.
            \item This is very convenient, since the logic output of most microcontrollers (e.g. Arduino) is $5V$, so no step-down is needed.
            \item However, because of the voltage bounds mentioned above, this means we can't send zero-offset AC signals (like we're working with here) through the digipot
            \item We could add an offset using an opamp, but that makes scaling very difficult, and adds unnecessary noise.
        \end{itemize}
    \subsection*{Bipolar Mode}
         Instead, we can operate in bipolar mode, where $V_{\text{SS}} < 0$. Some considerations in this mode:
        \begin{itemize}
            \item The potential bounds still apply. If we power the digipot with $\pm 2.5V$, the signal must fall within those bounds as well. This should not be a problem, since our signal should be within the $\pm1V$ envelope.
            \item The logic voltage must be within $0.3V$ of $V_\text{DD}$ (or specifically, when $V_\text{DD} = 3V$, the logic must be between $2.6V$ and $3.3V$, as per the datasheet).
            \item Logic low is still ground.
            \item There are $3.3V$ logic Arduinos, which could directly drive the chip in this mode. I am looking to acquire one of these, to eliminate the three voltage dividers needed with a $5V$ arduino.
        \end{itemize}
    \subsection*{Communication}
        \begin{itemize}
            \item We communicate with the potentiometer using SPI (serial peripheral interface).
            \item To set a resistance:
            \begin{itemize}
                \item Drive the CS (chip select) pin low
                \item Write the address as a single byte, MSB first. Because valid addresses are $0-5$ (zero-indexed), the word will look like \texttt{0b00000101} (for address 5).
                \item In Arduino, this can be done with \texttt{SPI.transfer(channel)}, where channel is in the range $0-5$.
                \item Then write the desired potentiometer value as a single byte, ranging from 0 (minimum resistance) to 255 (maximum resistance). In theory, the output resistance at step $n$ will be
                \begin{equation*}
                    R_\text{out} = (n/255)\cdot \SI{50}{\kilo\ohm}
                \end{equation*}
                Experimental response curves are detailed below.
                \item Drive the CS pin high.
                \item The code used on the Arduino is in the associated Github repository.
            \end{itemize}
        \end{itemize}
    \subsection*{Odds and ends}
        \begin{itemize}
            \item During testing, I used an external power supply tuned to $\pm2.5V$. However, to simplify the circuit (the less wires running on and off the board, the better), I am using voltage dividers off the $\pm15V$ opamp bus.
            \item Because of gaps in our resistor set, I was not able to power the chip at $\pm2.5V$. Instead, it is powered at $-2V/+3V$. This should not present a problem, since it still gives us the $\pm2V$ envelope, which should be sufficient. 
            \item IMPORTANT!!! The AD5206 comes in three varieties---$\SI{10}{\kilo\ohm}$, $\SI{50}{\kilo\ohm}$, and $\SI{100}{\kilo\ohm}$. The 10 and 100 varieties DO NOT respond properly in bipolar mode. This was not documented anywhere, but only the $\SI{50}{\kilo\ohm}$ works for our purposes.
            \item The only possible explanation I was able to find is that the $\SI{50}{\kilo\ohm}$ is listed as RoHS compliant on DigiKey, while the others are not. The $\SI{50}{\kilo\ohm}$ could therefore be a newer make. This stymied me for about 5 days, since I was using a $\SI{100}{\kilo\ohm}$.
        \end{itemize}
        \subsection*{Response curves}
        \begin{tabular}{lll}
            \toprule
            Potentiometer & Base resistance ($\Omega$) & Response ($\Omega/\text{tick}$) \\ \midrule
            Ideal & $50$\footnote{``Typical'' wiper resistance from datasheet} & $196.1$ \\
            R1 & $30.3$ & $202.6$ \\
            R2 & $39.5$ & $202.3$ \\
            R3 & $39.9$ & $202.1$ \\
            R4 & $42.0$ & $202.1$ \\
            R5 & $10.9$ & $202.6$ \\
            R6 & $27.4$ & $202.4$ \\\bottomrule
        \end{tabular}\vspace{1em}

        \noindent These resistors have a small (near-negligible) wiper resistance, and a slightly higher maximum resistance than advertised. The typical maximum resistance was about $\SI{51.5}{\kilo\ohm}$. No statistically significant nonlinearity was observed.
    
\end{document}